\documentclass{article}


\usepackage[a4paper]{geometry}

\usepackage{mathtools,amssymb}


\usepackage[T1,T2A]{fontenc}

\usepackage[utf8]{inputenc}

\usepackage[russian]{babel}

\usepackage[useregional]{datetime2}

\usepackage{listings}

\begin{document}
	
\begin{center}
	\begin{LARGE}
		\textbf{Язык C++}
	\end{LARGE}
\end{center}
\begin{center}
	\begin{normalsize}
		\textbf{Мещерин Илья}
	\end{normalsize}
\end{center}
\begin{center}
	\begin{Large}
		\textbf{Лекция 6}
	\end{Large}
\end{center}

\begin{center}
	\begin{large}
		\textbf{Наследование}
	\end{large}
\end{center}

\noindent \textbf{5.1) Объявление}\\
\textit{Derived} наследник \textit{Base}
\lstinputlisting[language=C++, firstnumber=1, tabsize=3,belowcaptionskip=5pt, columns=flexible, linerange={1-3}, xleftmargin=0.8cm]{code6.cpp}
\begin{itemize}
	\item \textit{public} - публичное наследование (все функции из вне класса знают о том, что класс является наследником и могут вызывать методы родителя, но, конечно, только те методы, которые у родителя являются публичными)
	\item \textit{private} - приватное наследование (сам класс знает от кого наследуется, но никто из вне не знает)
	\item \textit{protected} - protected наследование (сам класс знает от кого наследуется, а так же его потомки то же об этом знают)
\end{itemize}
\textbf{5.2) Поиск имен при наследовании}\\
\textbf{5.2a) Сокрытие имен наследником}
\lstinputlisting[language=C++, firstnumber=1, tabsize=3,belowcaptionskip=5pt, columns=flexible, linerange={5-18}, xleftmargin=0.8cm]{code6.cpp}
В данном случае вызовется версия функции \textit{f()}, описанная в классе \textit{Derived}, что логично\\
\lstinputlisting[language=C++, firstnumber=1, tabsize=3,belowcaptionskip=5pt, columns=flexible, linerange={20-33}, xleftmargin=0.8cm]{code6.cpp}
Ошибка компиляции. Правило поиска имен - имя ищется сначала в классе \textit{Derived} и все функции с таким именем участвуют в перегрузке и только они.



\end{document}
