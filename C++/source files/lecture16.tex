\documentclass{article}


\usepackage[a4paper]{geometry}

\usepackage{mathtools,amssymb}


\usepackage[T1,T2A]{fontenc}

\usepackage[utf8]{inputenc}

\usepackage[russian]{babel}

\usepackage[useregional]{datetime2}

\usepackage{listings}

\begin{document}

\newcommand{\cs}[1]{\lstinputlisting[language=C++, firstnumber=1, tabsize=3,belowcaptionskip=5pt, columns=flexible, linerange={#1}, xleftmargin=0.8cm]{code16.cpp}}
\newcommand{\cb}[1]{\lstinputlisting[language=C++, numbers=left, firstnumber=1, tabsize=3,belowcaptionskip=5pt, columns=flexible, linerange={#1}]{code16.cpp}}
\newcommand{\ti}{\textit}

\begin{center}
	\begin{LARGE}
		\textbf{Язык C++}
	\end{LARGE}
\end{center}
\begin{center}
	\begin{normalsize}
		\textbf{Мещерин Илья}
	\end{normalsize}
\end{center}
\begin{center}
	\begin{Large}
		\textbf{Лекция 16}
	\end{Large}
\end{center}	 

\noindent \textbf{10.6) Insert iterator}
\cs{1-2}
Копирует элементы диапазона [\ti{first, last}) в диапазон, начинающийся с \ti{d\_first}. 
\cs{4-5}
Методы \ti{operator*, operator++} ничего не делают.
\cs{7-9}
В данном случае произойдет \ti{push\_back()} в вектор.
\cs{12-12}
Принимает контейнер и возвращает \ti{std::back\_insert\_iterator} от него.
\begin{center}
	\begin{large}
		\textbf{Move-семантика и rvalue-ссылки}
	\end{large}
\end{center}
\noindent \textbf{11.1) Мотивировка}
\cs{14-22}
Хочется делать операции \ti{swap(), push\_back()} без лишних копирований\\
\noindent \textbf{11.2) std::move}
\cs{24-32}
Здесь эта проблема решается.
\newpage
\noindent \textbf{11.3) move-конструктор и move-assigment оператор}
\cs{34-37}
Сигнатуры конструктора перемещения и оператора перемещения.
В предыдущих примерах \ti{std::move} преобразует тип объекта в такой, чтобы вызывался конструктор перемещения. Конструктор перемещения создается по умолчанию, и отличается от конструктора копирования тем, что каждое поле инициализируется объектом, пропущенным через \ti{std::move}. Аналогично по умолчанию работает оператор перемещения. Если написать свой нетривиальный конструктор копирования, оператор присваивания или деструктор, то конструктор перемещения и оператор перемещения генерироваться компилятором не будет. В этом заключается \textbf{правило пяти}, т.е. если хоть какой-то из пяти озвученных ранее методов реализуется нетривиально, то и остальные, наверняка, реализоваться должны не по умолчанию.\\
\noindent \textbf{11.4) Реализация std::move}
\cs{39-42}
Пример реализации.
\cs{43-46}
Или вот так, начиная с \ti{C++14}.\\
Функция принимает объект именно такого типа, чтобы возможно было использовать ее и для lvalue объектов, и для rvalue объектов (если написать один амперсанд, то функция не сможет принимать временные объекты).\\
\noindent \textbf{11.5) rvalue ссылки}
\cs{48-49}
Ошибка компиляции. Такую ссылку можно связывать только с временными объектами.
\cs{50-51}
А так писать можно.
\cs{52-52}
Ошибка компиляции.
\cs{53-53}
Так уже можно.\\
При работе с \ti{const} логика остается прежней: все, что не \ti{const}, нельзя проинициализировать через \ti{const}, а инициализировать \ti{const} через не \ti{const} можно.\\\\\\\\
\noindent \textbf{11.6) виды value}
\cs{56-59}
\textbf{prvalue} (pure rvalue) - настоящие временные только что созданные объекты.\\
\textbf{xvalue} (expired value) - объекты, которые не являются временными объектами, но трактуются как таковые. (например, результат вызова \ti{std::move()} или каста \ti{static\_cast<type}\&\&\ti{>})\\
\textbf{lvalue} - именованные объекты, результат вызова функций \ti{operator++, operator=, operator*} и другое.\\
\textbf{rvalue} - результат вызова \ti{operator++(int)} и другое.\\
\textbf{glvalue} (generalize lvalue).\\
Хорошая эвристика, позволяющая отличать эти типы - у \ti{lvalue} объекта можно взять адрес. Если объект - именованная сущность, то это точно объект типа \ti{lvalue}.
\end{document}