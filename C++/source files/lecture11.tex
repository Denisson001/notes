\documentclass{article}


\usepackage[a4paper]{geometry}

\usepackage{mathtools,amssymb}


\usepackage[T1,T2A]{fontenc}

\usepackage[utf8]{inputenc}

\usepackage[russian]{babel}

\usepackage[useregional]{datetime2}

\usepackage{listings}

\begin{document}

\newcommand{\cs}[1]{\lstinputlisting[language=C++, firstnumber=1, tabsize=3,belowcaptionskip=5pt, columns=flexible, linerange={#1}, xleftmargin=0.8cm]{code11.cpp}}
\newcommand{\cb}[1]{\lstinputlisting[language=C++, numbers=left, firstnumber=1, tabsize=3,belowcaptionskip=5pt, columns=flexible, linerange={#1}]{code11.cpp}}

\begin{center}
	\begin{LARGE}
		\textbf{Язык C++}
	\end{LARGE}
\end{center}
\begin{center}
	\begin{normalsize}
		\textbf{Мещерин Илья}
	\end{normalsize}
\end{center}
\begin{center}
	\begin{Large}
		\textbf{Лекция 11}
	\end{Large}
\end{center}	 


\noindent \textbf{7.3) Последовательность и правила обработки исключений (продолжение)}
\cs{1-6}
В данном случае компилятор запустит исключение, которое обрабатывалось в этом блоке, в дальнейший полет, но поймат его может только \textit{catch} уровнем выше, то есть все блоки \textit{catch}, написанные ниже, рассматриваться все равно не будут. 	
\cs{8-13}
Но так же можно кинуть и другой новый объект.\\
\noindent \textbf{7.4) Копирование при исключениях}
\cs{15-29}
Когда делаем \textit{throw} создается копия объекта и вызывается конструктор копирования. Это нужно, т.к. при выходе из блока все локальные объекты уничтожаются.\\\\\\\\\\
\cs{31-36}
Если в блоке \textit{catch} принимать по значению, а не по ссылке, то объект будет сконструирован суммарно 3 раза.
\cs{38-49}
В данном случае все упадет, потому что \textit{dynamic\_cast} кинет исключение \textit{std::bad\_cast}. Потому что \textit{f} уже не будет являться ссылкой на класс \textit{Derived}, т.к. при вызове \textit{throw} компилятор смотрит на статический тип объекта и создает его копию.
\cs{51-57}
В данном случае \textit{dynamic\_cast} отработает правильно, но дальше полетит объект типа \textit{Base} и снова прикастовать его к типу \textit{Derived} не получится.
\cs{58-60}
А если написать так, то копирования (и соответственно срезки при копировании) не произойдет и можно будет кастовать дальше этот объект к типу \textit{Derived}.\\
\noindent \textbf{7.5) Спецификация исключений}\\
\noindent \textbf{7.5а) Old version}
\cs{62-62}
В круглых скобках можно перечислить все типы объектов, которые может кинуть функция.\\
Но если все-таки кинуть объект типа, который не был указан, то сразу возникнет runtime error. \\\\\\\\
\noindent \textbf{7.5б) New version}
\cs{103-103}
В данном случае слово \textit{noexcept} будет являться \textbf{спецификатором}, которое означает, что если эта функция внутри себя сделает \textit{throw}, то вызовется функция \textit{std::terminate} и программа завершится. Но если внутри функции отловить все исключения, то все ОК.\\
Слово \textit{noexcept()} может быть булевским оператором, который возвращает \textit{true}, если то, что написано в скобках является выражением потенциально бросающим исключения (например, там есть \textit{throw}, \textit{new}, \textit{dynamic\_cast} или вызов функции, которая не \textit{noexcept}). 
\cs{105-117}
В примере рассматривается случай, когда, чтобы узнать является ли функция \textit{noexcept} или нет, нужно знать являются или другие вызываемые внутри функции \textit{noexcept}. Писать \textit{noexcept} нужно 2 раза, т.к. после первого раза мы определяем, является ли функция \textit{g(x)} \textit{noexcept} или нет, а второй раз для того чтобы сказать является ли сама функция \textit{f()} \textit{noexcept} или нет.
\cs{119-124}
В данном примере оператор \textit{noexcept()} должен еще на этапе компиляции знать результат выполнения функции \textit{fibonacci(10)}. Поэтому это просто не скомпилируется.
\cs{126-126}
А вот так это уже скомпилируется. С помощью слова \textit{constexpr} можно создавать переменные, функции и даже объекты, которые будут рассчитаны на этапе компиляции.\\\\\\\\\\\\\\\\\\
\noindent \textbf{7.6) Исключения в конструкторах}
\cs{64-79}
В данном случае произойдет утечка памяти, т.к. деструктор не вызовется (он не может вызваться, т.к. если, например, после момента возникновения исключения в конструкторе в коде захватывалась память, то в деструкторе она должна освободиться, хотя и не была выделена, и программа бы сразу падала). Поля класса деструктурируются, т.к. они уже сгенерированы на момент входа в конструктор, но захваченная память не освободится.
\cs{81-83}
Пример потенциальной утечки памяти.\\
\noindent \textbf{7.7) Исключения в деструкторах}
\cs{85-95}
Главная проблема с исключениями в деструкторах связана с тем, что деструктор может вызваться при обработке исключения, потому что два исключения одновременно не могут поддерживаться. 
\cs{97-97}
Функция, которая возвращает \textit{true}, если в данный момент летит исключение.\\
\cs{99-99}
Функция, которая используется в ситуациях, когда необходимо аварийно завершить программу. Эта функция по умолчанию вызывает функцию \textit{abort()}, которая выполняет необходимые системные вызовы для завершения программы.
\cs{101-101}
Можно передать этой функции указатель на функцию или функциональный объект, которая должна вызываться вместо \textit{std::terminate}. По стандарту эта функция в конечном итоге все равно должна завершать программу. \\
\noindent \textbf{7.8) Гарантия безопасности при исключениях}\\
\textbf{Нестрогая гарантия безопасности}: если в ходе выполнения какой-то операции вылетает исключение, то, по крайней мере, все объекты остаются в валидном состоянии, не происходит утечки ресурсов, не происходит undefined behavior.\\
\textbf{Cтрогая гарантия безопасности}: если в ходе выполнения какой-то операции вылетает исключение, то объект остается в том состоянии, в котором он был изначально. Например, если в \textit{std::vector} во время выполнения операции \textit{push\_back()}, а точнее при копировании, бросится исключение, то вся структура объекта вернется в то состояние, которое было до выполнения операции.\\
(STL-контейнеры дают строгую гарантию безопасности.)\\
\noindent \textbf{7.9) Стандартные исключения}\\
Существует класс \textit{std::exception}. Все исключения, генерируемые стандартной библиотекой наследуются от него. У них всех есть метод \textit{what()}, который возвращает строку, в которой написано, что случилось.
\end{document}