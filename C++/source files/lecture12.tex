\documentclass{article}


\usepackage[a4paper]{geometry}

\usepackage{mathtools,amssymb}


\usepackage[T1,T2A]{fontenc}

\usepackage[utf8]{inputenc}

\usepackage[russian]{babel}

\usepackage[useregional]{datetime2}

\usepackage{listings}

\begin{document}
	
\begin{center}
	\begin{LARGE}
		\textbf{Язык C++}
	\end{LARGE}
\end{center}
\begin{center}
	\begin{normalsize}
		\textbf{Мещерин Илья}
	\end{normalsize}
\end{center}
\begin{center}
	\begin{Large}
		\textbf{Лекция 12}
	\end{Large}
\end{center}
\begin{center}
	\begin{large}
		\textbf{Аллокаторы}
	\end{large}
\end{center}

\noindent \textbf{8.1) Placement new} \\
Явный вызов конструктора и деструктора на месте
\lstinputlisting[language=C++, numbers=left, firstnumber=1, tabsize=3,belowcaptionskip=5pt, columns=flexible, linerange={1-3}]{code12.cpp}
\noindent \textbf{8.2) Перегрузка операторов new, delete}\\
\textit{n} - кол-во байт
\lstinputlisting[language=C++, firstnumber=1, tabsize=3,belowcaptionskip=5pt, columns=flexible, linerange={5-5}, xleftmargin=0.8cm]{code12.cpp}
При вызове оператора \textit{new}
\begin{itemize}
	\item[а)]Выделяется память
	\item[б)]Вызывается конструктор
\end{itemize}
Когда перегружаем \textit{new} думаем только о первой части, вторую часть перегрузить нельзя\\
Оператор \textit{new} и функция \textit{new()} - разные вещи\\\\
Реализация перегрузки функции \textit{new} (т.к. пункт б делается сам)
\lstinputlisting[language=C++, firstnumber=1, tabsize=3,belowcaptionskip=5pt, columns=flexible, linerange={7-7}, xleftmargin=0.8cm]{code12.cpp}
В оператор \textit{new} можно передать дополнительные аргументы
\lstinputlisting[language=C++, firstnumber=1, tabsize=3,belowcaptionskip=5pt, columns=flexible, linerange={9-10}, xleftmargin=0.8cm]{code12.cpp} 
При вызове оператора \textit{delete}
\begin{itemize}
	\item[а)]Вызывается деструктор
	\item[б)]Освобождается память
\end{itemize}
Только вторую часть можно перегрузить\\
Можно написать \textit{delete(...)} но вызывать эту функцию нужно руками, то есть сначала вызвать деструктор \textit{p->\textasciitilde T()} потом наш \textit{delete(...)}
(стандартный \textit{delete p} переопределить с параметрами нельзя)\\\\\\\\\\\\
Пример
\lstinputlisting[language=C++, firstnumber=1, tabsize=3,belowcaptionskip=5pt, columns=flexible, linerange={12-18}, xleftmargin=0.8cm]{code12.cpp}
Все то же самое отдельно для операторов \textit{new[]} и \textit{delete[]}\\\\
Совет: если реализовали кастомный \textit{new}, то стоит реализовать \textit{delete} с такими же параметрами такой \textit{delete} вызовется, если конструктор кинет исключение\\\\
Пример
\lstinputlisting[language=C++, numbers=left, firstnumber=1, tabsize=3,belowcaptionskip=5pt, columns=flexible, linerange={20-46}]{code12.cpp}
Вызовется правильный \textit{delete} не смотря на то, что \textit{static operator delete}\\\\\\\\\\
\noindent \textbf{8.3) Nothrow new}
\lstinputlisting[language=C++, firstnumber=1, tabsize=3,belowcaptionskip=5pt, columns=flexible, linerange={48-52}, xleftmargin=0.8cm]{code12.cpp}
Такой вызов оператора \textit{new} не бросает исключений\\\\
\noindent \textbf{8.4) New\_handler}\\
Правильная реализация функции \textit{new}: в бесконечном цикле пытаемся выделить память и если не смогли, то вызываем \textit{new\_handler()}\\
То есть эта функции по логике должна каким-то образом помогать функции \textit{new} выделить память
\lstinputlisting[language=C++, firstnumber=1, tabsize=3,belowcaptionskip=5pt, columns=flexible, linerange={54-55}, xleftmargin=0.8cm]{code12.cpp}
В \textit{std::set\_new\_handler()} можно передать свою функцию \textit{new\_handler()}, по умолчанию \textit{nullptr}\\

\noindent \textbf{8.5) Реализация аллокатора}
\lstinputlisting[language=C++, numbers=left, firstnumber=1, tabsize=3,belowcaptionskip=5pt, columns=flexible, linerange={57-72}]{code12.cpp}
Аллокатор - обертка над \textit{new} и \textit{delete}\\
\noindent \textbf{8.6) std::allocator\_traits}
\lstinputlisting[language=C++, firstnumber=1, tabsize=3,belowcaptionskip=5pt, columns=flexible, linerange={75-76}, xleftmargin=0.8cm]{code12.cpp}
Реализует за аллокатор те методы, которые аллокатор не реализовал\\
\textit{construct()} и \textit{destroy()} у всех аллокаторов реализуется одинаково, поэтому эту реализацию можно вынести в отдельный класс - \textit{allocator\_traits}\\
Вектор реализуется через обращения к \textit{allocator\_traits}\\
\lstinputlisting[language=C++, firstnumber=1, tabsize=3,belowcaptionskip=5pt, columns=flexible, linerange={78-83}, xleftmargin=0.8cm]{code12.cpp}
Аналогично реализуется \textit{deallocate()}\\
Для \textit{construct(), destroy()} класс умеет проверять реализованы ли уже эти функции в аллокаторе с нужной сигнатурой и иначе реализует сам
\end{document}