\documentclass{article}


\usepackage[a4paper]{geometry}

\usepackage{mathtools,amssymb}


\usepackage[T1,T2A]{fontenc}

\usepackage[utf8]{inputenc}

\usepackage[russian]{babel}

\usepackage[useregional]{datetime2}

\usepackage{listings}

\title{Язык C++. Лекция 13}

\author{Мещерин Илья}

\date{\DTMdate{2018-12-10}}


\begin{document}

\maketitle

\textbf{8.7) Пример нестандартного аллокатора} \\
\textit{StackAllocator<int>} \\
Сначала выделяет большой кусок памяти. Особенность в том, что при deallocate ничего не делает (экономит время работы). 

\begin{center}
	\begin{large}
		\textbf{Контейнеры}
	\end{large}
\end{center}


\noindent \textit{vector, list, deque} - Sequence containers (реализация идеи последовательного хранения элементов)\\
\textit{map, set (+multi, unordered)} - Associative containers (реализация идеи построения отображения)\\
\textbf{9.1) std::vector}
\begin{itemize}
\lstinputlisting[language=C++, numbers=left, firstnumber=1, tabsize=3,belowcaptionskip=5pt, columns=flexible, linerange={1-13}]{code13.cpp}
	\item[а)] 
\textit{push\_back(), pop\_back(), emplace\_back()} (создает объект непосредственно в конце вектора из аргументов)\\
	\lstinputlisting[language=C++, numbers=left, firstnumber=1, tabsize=3,belowcaptionskip=5pt, columns=flexible, linerange={15-36}]{code13.cpp}
	\item[б)] Оператор {[ ]}\textit{, at()} (если вышли за границы, то кинет исключение, при использовании [ ] будет ub)
\lstinputlisting[language=C++, numbers=left, firstnumber=1, tabsize=3,belowcaptionskip=5pt, columns=flexible, linerange={38-42}]{code13.cpp}
	\item[в)] Конструктор, деструктор, оператор =\\
При копировании нужно определить аллокатор для нового контейнера. Функция ниже возвращает аллокатор для нового объекта.
\lstinputlisting[language=C++, tabsize=3,belowcaptionskip=5pt, columns=flexible, linerange={44-44}]{code13.cpp}
	\item[г)] \textit{swap()}\\
Меняет местами указатели на массивы и меняет местами аллокаторы.
	\item[д)] \textit{size(), capacity(), front(), back(), resize(), shrink\_to\_fit()} (делает размер вектора равным числу элементов в нем)
\end{itemize}

\noindentРазмер вектора константа
\lstinputlisting[language=C++, numbers=left, firstnumber=1, tabsize=3,belowcaptionskip=5pt, columns=flexible, linerange={47-50}]{code13.cpp}

\noindentУпадет из-за двойного удаления по одному месту.
\lstinputlisting[language=C++, numbers=left, firstnumber=1, tabsize=3,belowcaptionskip=5pt, columns=flexible, linerange={52-55}]{code13.cpp}



\end{document}