\documentclass{article}


\usepackage[a4paper]{geometry}

\usepackage{mathtools,amssymb}


\usepackage[T1,T2A]{fontenc}

\usepackage[utf8]{inputenc}

\usepackage[russian]{babel}

\usepackage[useregional]{datetime2}

\usepackage{listings}

\begin{document}

\newcommand{\cs}[1]{\lstinputlisting[language=C++, firstnumber=1, tabsize=3,belowcaptionskip=5pt, columns=flexible, linerange={#1}, xleftmargin=0.8cm]{code17.cpp}}
\newcommand{\cb}[1]{\lstinputlisting[language=C++, numbers=left, firstnumber=1, tabsize=3,belowcaptionskip=5pt, columns=flexible, linerange={#1}]{code17.cpp}}
\newcommand{\ti}{\textit}

\begin{center}
	\begin{LARGE}
		\textbf{Язык C++}
	\end{LARGE}
\end{center}
\begin{center}
	\begin{normalsize}
		\textbf{Мещерин Илья}
	\end{normalsize}
\end{center}
\begin{center}
	\begin{Large}
		\textbf{Лекция 17}
	\end{Large}
\end{center}	 

\noindent \textbf{11.6) Виды value}
\cs{1-2}
В во второй реализации не будет происходить лишнего копирования.\\
\noindent \textbf{11.7) Универсальные ссылки}
\cs{4-5}
Если функция имеет такую сигнатуру (как у \ti{std::move()}), то, вызвав ее от \ti{lvalue} объекта типа \ti{type}, вместо \ti{T} подставится \ti{type}\& (да, это костыль).\\
\noindent \textbf{11.7$\frac{1}{2}$) Reference collapsing (сворачивание ссылок)}\\
Если в процессе раскрытия шаблонных подстановок могут возникнуть следующие ситуации:\\
\ti{type}\&\& \&  --> \ti{type}\&\\
\ti{type}\& \&\&  --> \ti{type}\&\\
\ti{type}\&\& \&\&  --> \ti{type}\&\&
\cs{2-2}
Это \textbf{не будет} универсальной ссылкой, т.к. \ti{T} - шаблонный параметр класса.
\cs{7-8}
Также \textbf{не является} универсальной ссылкой.\\
\noindent \textbf{11.8) std::forward и прямая передача}\\
Функция \ti{std::forward()}, в отличие от \ti{std::move()}, кастует объект к \ti{rvalue} тогда и только тогда, когда приняла его по \ti{rvalue} ссылке.
\cs{10-13}
Правильная реализация \ti{construct()} в аллокаторе (если объекты были типа \ti{lvalue}, а при, например, использовании \ti{std::move()}, вместо \ti{std::forward()}, они стали бы типа \ti{rvalue} и могли бы испортиться где-то дальше).\\ Похожим образом нужно изменить реализацию метода \ti{emplace\_back()} у вектора.\\
\noindent \textbf{11.9) Реализация std::forward}
\cs{15-18}
\cs{29-33}
Вторая реализация отдельно, чтобы принимать \ti{rvalue} объекты.\\
\noindent \textbf{11.10) std::move\_if\_noexcept}\\
Теперь стало понятно, что при реаллокации в векторе нужно делать перемещение, а не копирование. Но что если при перемещении вылетело исключение? Стандартная библиотека гарантирует безопасность относительно исключений. Но в нашей ситуации один массив испорчен, а другой не до конца создан. Делать перемещения обратно делать нельзя, т.к. может опять вылететь исключение. Решение - делать перемещение, только если конструктор перемещения является \ti{noexcept}, в противном случае нужно делать копирование. Функция \ti{std::move\_if\_noexcept} делает каст к \ti{rvalue} если конструктор перемещения является \ti{noexcept}. Реализация этой функции откладывается на несколько лекций.\\
\noindent \textbf{11.11) Return Value Optimization}
\cs{20-24}
В данном случае объект \ti{ans} создается на том месте памяти, куда должен быть возвращен из функции, т.е. промежуточного объекта не создается. Из-за сложной логики функции компилятор не всегда может сделать такой трюк, в таком случае он сделает перемещение.\\
\noindent \textbf{11.12) Reference qualifiers}
\cs{26-27}
Если это методы некоторого класса, то, в зависимости от того является объект этого класса в данный момент \ti{rvalue} или \ti{lvalue}, вызовутся разные версии этого этого метода. В частности можно таким способом запретить конструкцию вида \ti{a + b = c}.
\end{document}