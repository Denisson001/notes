\documentclass{article}


\usepackage[a4paper]{geometry}

\usepackage{mathtools,amssymb}


\usepackage[T1,T2A]{fontenc}

\usepackage[utf8]{inputenc}

\usepackage[russian]{babel}

\usepackage[useregional]{datetime2}

\usepackage{listings}

\title{Язык C++. Лекция 7}

\author{Мещерин Илья}

\date{\DTMdate{2018-10-29}}


\begin{document}

\maketitle

\noindent \textbf{5.2б) Явный вызов методов предка}
\lstinputlisting[language=C++, firstnumber=1, tabsize=3,belowcaptionskip=5pt, columns=flexible, linerange={1-12}, xleftmargin=0.8cm]{code7.cpp}
Явно пишем откуда взять нужную функцию. Если наследование приватное, то все равно ошибка компиляции.
\lstinputlisting[language=C++, firstnumber=1, tabsize=3,belowcaptionskip=5pt, columns=flexible, linerange={14-17}, xleftmargin=0.8cm]{code7.cpp}
Второй способ решения вопроса. В таком случае вызывать функцию \textit{f()} можно как обычно.
\noindent \textbf{5.2в) Пример}\\
Оффтоп - если при наследовании не написать тип наследования, то по умолчанию \textit{private} (у \textit{struct} он \textit{public})
\lstinputlisting[language=C++, firstnumber=1, tabsize=3,belowcaptionskip=5pt, columns=flexible, linerange={19-32}, xleftmargin=0.8cm]{code7.cpp}
Ошибка компиляции.\begin{large}
	\textbf{Проверка доступа происходит после поиска имен.}
\end{large}
\noindent \textbf{5.2г) Пример}
\lstinputlisting[language=C++, firstnumber=1, tabsize=3,belowcaptionskip=5pt, columns=flexible, linerange={34-47}, xleftmargin=0.8cm]{code7.cpp}
Ошибка компиляции. Название типа \textit{Granny} запрещено внутри класса \textit{Son}.
\lstinputlisting[language=C++, firstnumber=1, tabsize=3,belowcaptionskip=5pt, columns=flexible, linerange={49-54}, xleftmargin=0.8cm]{code7.cpp}
А так уже писать можно, т.к. в глобальной области видимости запретов нет.
\lstinputlisting[language=C++, firstnumber=1, tabsize=3,belowcaptionskip=5pt, columns=flexible, linerange={56-59}, xleftmargin=0.8cm]{code7.cpp}
Все равно ошибка компиляции, т.к. ошибка в выражении \textit{s.Granny::a} возникает после точки, а не после двух двоеточий.
\lstinputlisting[language=C++, firstnumber=1, tabsize=3,belowcaptionskip=5pt, columns=flexible, linerange={61-64}, xleftmargin=0.8cm]{code7.cpp}
Такой способ уже решает проблему.\\
\noindent \textbf{5.2д) Пример}
\lstinputlisting[language=C++, firstnumber=1, tabsize=3,belowcaptionskip=5pt, columns=flexible, linerange={66-79}, xleftmargin=0.8cm]{code7.cpp}
Все ОК, но вообще отношение дружбы не транзитивно.\\
\noindent \textbf{5.3) Порядок вызова конструкторов и деструкторов при наследовании}

\end{document}