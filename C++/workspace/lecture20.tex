\documentclass{article}


\usepackage[a4paper]{geometry}

\usepackage{mathtools,amssymb}


\usepackage[T1,T2A]{fontenc}

\usepackage[utf8]{inputenc}

\usepackage[russian]{babel}

\usepackage[useregional]{datetime2}

\usepackage{listings}

\begin{document}

\newcommand{\cs}[1]{\lstinputlisting[language=C++, firstnumber=1, tabsize=3,belowcaptionskip=5pt, columns=flexible, linerange={#1}, xleftmargin=0.8cm]{code19.cpp}}
\newcommand{\cb}[1]{\lstinputlisting[language=C++, numbers=left, firstnumber=1, tabsize=3,belowcaptionskip=5pt, columns=flexible, linerange={#1}]{code19.cpp}}
\newcommand{\ti}{\textit}

\begin{center}
	\begin{LARGE}
		\textbf{Язык C++}
	\end{LARGE}
\end{center}
\begin{center}
	\begin{normalsize}
		\textbf{Мещерин Илья}
	\end{normalsize}
\end{center}
\begin{center}
	\begin{Large}
		\textbf{Лекция 19}
	\end{Large}
\end{center}	 

\begin{center}
	\begin{large}
		\textbf{Умные указатели}
	\end{large}
\end{center}

\noindent \textbf{12.6) Weak\_ptr}\\
Такой умный указатель предназначен для разрешения циклов в указателях.
\cs{1-2}
Функция \ti{expired()} проверяет, был ли удален объект, на который ссылается ti{weak\_ptr}. Функция \ti{lock()} создает \ti{shared\_ptr}, который управляет объектом, на который ссылается \ti{weak\_ptr}.\\
Из \ti{unique\_ptr} можно конструировать \ti{shared\_ptr}, а, наоборот, нельзя.\\
\noindent \textbf{12.7) Deleter}\\
Умных указатели идеологически помогают решить не только проблему с утечками памяти. С их помощью можно поддерживать, например, открытый файл или сетевое соединение. Достаточно передать свой кастомный \ti{Deleter}.
\cs{4-7}
Такие параметры имеет класс \ti{unique\_ptr} на самом деле.
\cs{9-11}
Примерно это делает \ti{std::default\_delete<T>}.\\
Функции \ti{make\_unique()} и \ti{make\_shared()} не поддерживают эту фичу.\\
\noindent \textbf{12.8) std::enable\_shared\_from\_this}
\cs{13-16}
Последняя функция возвращает \ti{shared\_ptr} на \ti{this}. Это является решением проблемы создания \ti{shared\_ptr} от одного и того же указателя (\ti{this}) несколько раз (что делать нельзя).\\
Примерная реализация: внутри класса \ti{std::enable\_shared\_from\_this} хранится \ti{weak\_ptr} на \ti{this}, который делает \ti{lock()} при необходимости (тут все сложнее, лучше погуглить).
\begin{center}
	\begin{large}
		\textbf{Вывод типов (type deduction)}
	\end{large}
\end{center}
\noindent \textbf{12.1) Range-based for loop}
\cs{18-18}
Пробег по контейнеру (\ti{std::map}) без использования итераторов.
\cs{19-19}
Так писать плохо, т.к. происходит копирование. На самом деле в первом случае тоже произойдет копирование, т.к. в \ti{std::mape} объектом является \ti{std::pair<const MyType, std::string>}, поэтому произойдет каст и копирование.\\
\noindent \textbf{12.2) Auto}
\cs{21-21}
Тут компилятор может сам определить тип \ti{x}.	
\cs{23-23}
Удобный \ti{for}.
\cs{25-26} 
Подстановка типа идет по тем же правилам, как и в случае шаблонного вывода.
\cs{28-28}
Второй и последний случай, когда ссылка является универсальной.
\cs{30-30}
Тут универсальность теряется.\\
\noindent \textbf{12.3) std::initializer\_list}
\cs{32-34}
В последнем случае \ti{x} имеет тип \ti{std::initializer\_list}.
\end{document}