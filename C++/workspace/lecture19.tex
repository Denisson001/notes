\documentclass{article}


\usepackage[a4paper]{geometry}

\usepackage{mathtools,amssymb}


\usepackage[T1,T2A]{fontenc}

\usepackage[utf8]{inputenc}

\usepackage[russian]{babel}

\usepackage[useregional]{datetime2}

\usepackage{listings}

\begin{document}

\newcommand{\cs}[1]{\lstinputlisting[language=C++, firstnumber=1, tabsize=3,belowcaptionskip=5pt, columns=flexible, linerange={#1}, xleftmargin=0.8cm]{code18.cpp}}
\newcommand{\cb}[1]{\lstinputlisting[language=C++, numbers=left, firstnumber=1, tabsize=3,belowcaptionskip=5pt, columns=flexible, linerange={#1}]{code18.cpp}}
\newcommand{\ti}{\textit}

\begin{center}
	\begin{LARGE}
		\textbf{Язык C++}
	\end{LARGE}
\end{center}
\begin{center}
	\begin{normalsize}
		\textbf{Мещерин Илья}
	\end{normalsize}
\end{center}
\begin{center}
	\begin{Large}
		\textbf{Лекция 18}
	\end{Large}
\end{center}	 

\begin{center}
	\begin{large}
		\textbf{Умные указатели}
	\end{large}
\end{center}

\noindent \textbf{12.1) Мотивировка}
\cs{1-5}
Если функция \ti{g()} бросает исключение, то возможна утечка памяти.\\
\noindent \textbf{12.2) Auto\_ptr}\\
Класс представляет скорее историческую ценность, а не практическую.
\cs{7-13}
Объекты такого класса нельзя копировать и присваивать, то есть, например, нельзя создать вектор объектов такого класса.\\
\noindent \textbf{12.3) Unique\_ptr}\\
Класс реализует концепцию единоличного владения
\cs{15-31}
Проблема предыдущего класса решается с помощью move-семантики. Тут может возникнуть проблема, если тип \ti{T} является, например, массивом, тогда реализация не является корректной.
\cs{33-33}
Но в стандартной библиотеке написана специализация для решения этой проблемы, но вообще не рекомендуется создавать \ti{unique\_ptr} на массив, лучше перед этим обернуть массив в контейнер.\\
Еще нужно определить \ti{operator*, operator->}, а у \ti{unique\_ptr} от массива только \ti{operator}[].
\cs{35-39}
Так решается исходная проблема, поставленная в начале лекции.\\
\noindent \textbf{12.4) Shared\_ptr}\\
Нужно научиться создавать такой умный указатель, который мог бы правильно обрабатывать ситуацию с хранением обычного указателя на один и тот же объект в разных умных указателях. 
\cs{41-55}
Такой способ решения проблемы запрещает создавать несколько \ti{shared\_ptr} от одного и того же обычного указателя. Оператор и конструктор копирования должны увеличивать счетчик внутри структуры \ti{Helper} на один, оператор и конструктор перемещения забирают объект типа \ti{Helper} себе, деструктор уменьшает счетчик внутри структуры \ti{Helper} на один, когда счетчик обнуляется происходит освобождение обычного указателя.\\
\noindent \textbf{12.5) Make\_shared, make\_unique}
\cs{57-57}
Способ использования.
\cs{59-62}
Реализация функции \ti{make\_shared()}.\\\\
\cs{64-64}
При таком вызове функции \ti{f()} сначала может создаться указатель на \ti{int}, затем выполниться функция \ti{g()}, которая может бросить исключение, а только потом сконструироваться \ti{shared\_ptr}, т.е. произойдет утечка памяти.
\cs{66-67}
Если реализовать функцию \ti{make\_shared()} таким образом, то можно сэкономить на одном вызове оператора \ti{new} и одном вызове оператора \ti{delete}, т.к. объект и счетчик лежат рядом в памяти.\\
\noindent \textbf{12.5$\frac{1}{2}$) Allocate\_shared}
\cs{69-70}
Таким образом можно выделять память не через оператор \ti{new}, а с помощью своего аллокатора.
\cs{72-76}
Начало реализации функции \ti{allocate\_shared()}.
\end{document}